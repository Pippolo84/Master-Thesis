\documentclass[a4paper, 11pt, oneside]{article}
\usepackage[utf8]{inputenc}
\usepackage[italian]{babel}
\usepackage{varioref}
\usepackage[pdftex,bookmarks=true,hypertexnames=true]{hyperref}
\hypersetup{
pdfauthor= {Juri Lelli},
pdftitle= {Analisi e miglioramento dell'efficienza di algoritmi di scheduling
basati su deadline per architetture multiprocessore},
pdfkeywords = {realtime, scheduling, linux, edf, multiprocessor, deadline},
pdfborder = { 0 0 0 0 }
}

\title{Analisi e miglioramento dell'efficienza di algoritmi di scheduling
real-time per architetture multiprocessore}
\author{Juri Lelli}

\begin{document}

\maketitle

\section*{Introduzione e scenario di applicazione}
I sistemi multiprocessore trovano al giorno d'oggi largo utilizzo nelle più
diverse configurazioni hardware, dalle server workstations ai dispositivi
mobili di ultima generazione. La sempre crescente domanda in termini di
miglioramento delle performance non trova infatti risposta in tecniche
volte all'aumento della capacità computazionale del singolo
processore~\cite{ARM}. Si
cercano invece soluzioni distribuite, ad esempio attraverso applicazioni
multithread, che, suddividendo il carico tra le CPU del sistema, riescano a
compiere il lavoro richiesto in minor tempo e con un maggior risparmio
energetico. 

I moderni sistemi operativi devono, dal canto loro, essere in grado di fornire
alle applicazioni gli strumenti per decidere in che modo suddividere le
attività ed i meccanismi per controllarne il funzionamento. Oltre a ciò è sempre
più importante poter coniugare le esigenze dei sistemi in tempo reale
(\emph{real-time systems}), caratterizzate da vincoli temporali, ed usi più generali,
come quelli dei sistemi \emph{general-purpose}. Analizzando le prime, è poi
possibile suddividere i sistemi in tempo reale in \emph{hard
real-time}, in cui il mancato rispetto dei vincoli temporali può comportare
il verificarsi di
eventi disastrosi per il sistema, e \emph{soft real-time}, in cui tale fenomeno
non comporta il verificarsi di situazioni critiche, ma solo il degrado della
qualità del servizio (QoS).

La fusione tra queste tendenze trova terreno fertile nel mondo dei sistemi
operativi il cui codice sorgente viene rilasciato dagli autori, e dalla
comunità di sviluppo, come ``Open/Free Software''. La libera e gratuita
disponibilità del codice e la possibilità di apportarvi modifiche ha reso
interessante, sia per la comunità scientifica che per le aziende, provare
ad inserire, all'interno di sistemi tradizionalmente general-purpose, elementi
e meccanismi tipici di quelli real-time. Tale approccio è particolarmente
interessante soprattutto per i sistemi soft real-time, in cui processi
real-time e non real-time devono spesso poter coesistere ed essere
gestiti dinamicamente all'interno dello stesso sistema operativo.

L'utilizzo sempre più diffuso del noto sistema operativo libero GNU/Linux (e
nello specifico del suo kernel \emph{Linux}) sulle più diverse piattaforme
hardware comporta la necessità di adattare un sistema nato come \emph{General
Purpose Operating System} (GPOS) ad ambienti che spesso richiedo l'utilizzo
di un \emph{Real-Time Operating System} (RTOS). Quando infatti (ad esempio
nel caso di telefoni cellulari o smart-phone su cui eseguono applicazioni
\emph{time-sensitive} come lettori audio/video) dimensioni,
capacità computazionale, consumo di energia e costi sono particolarmente
stringenti, bisogna necessariamente fare un uso efficiente delle risorse del
sistema e, al tempo stesso, riuscire a rispettare i vincoli real-time.

Sul mercato è da tempo possibile trovare versioni modificate del kernel Linux
con supporto ai sistemi in tempo reale~\cite{windriver, timesys, montavista}
, ma queste sono spesso non aperte e non
possono quindi trarre vantaggio dalla grande comunità di sviluppo dello
standard kernel. Recentemente è stata invece sviluppata, presso il  Real-Time
System Laboratory (ReTiS Lab~\cite{RETIS}) della Scuola Superiore Sant'Anna di
Pisa, un'implementazione dell'algoritmo ``Earliest Deadline First''
(EDF~\cite{LiuL1973, stankovic98}) all'interno
del kernel Linux~\cite{SCHED_EDF09, SCHEDDEAD, scheddeadv2}. EDF è un noto
algoritmo di scheduling real-time a priorità
dinamica basato su un concetto semplice: il processo con \emph{deadline}
(vincolo temporale che identifica l'istante di tempo entro il quale l'istanza
corrente deve terminare) più imminente è quello che deve eseguire per primo.
Si riesce inoltre ad assicurare l'isolamento temporale tra processi (ogni
applicazione presente nel sistema che richieda garanzie di tipo soft real-time
deve avere a disposizione una partizione delle risorse disponibili) utilizzando,
congiuntamente ad EDF, anche l'algoritmo ``Constant Bandwidth Server''
(CBS~\cite{abeniButtazzo98, abeniButtazzo04}).

Partendo da una implementazione pienamente utilizzabile solo su sistemi
monoprocessore, si è giunti poi ad estendere tale implementazione a sistemi
multiprocessore. Tale modifica è fondamentale perché permette di controllare
finemente l'allocazione delle risorse del sistema e rende possibile una gestione 
efficace dell'ambiente durante il funzionamento dinamico. Senza tale estensione
esistono inoltre situazioni in cui le risorse del sistema potrebbero rimanere
sotto utilizzate (si veda~\cite{HOS} per alcuni esempi) andando ad incidere
negativamente sul rapporto
versatilità del sistema/costo dell'hardware, dove con \emph{versatilità} si
intende la capacità del sistema di risolvere problemi diversi mediante lo
stesso hardware.

\section*{Obiettivi}
Il lavoro di tesi, che ha reso possibile l'utilizzo su sistemi multiprocessore
dell'algoritmo di scheduling real-time EDF/CBS, è tutt'altro che concluso.
Utilizzando l'implementazione corrente è possibile decidere su quanti e quali
processori del sistema un processo debba eseguire. In letteratura si parla di
scheduling \emph{completamente partizionato} quando si associa staticamente ad
ogni processo un solo processore tra quelli disponibili, contrariamente a quello
che avviene utilizzando scheduling di tipo \emph{globale}, dove tutti i processi
possono eseguire su tutti i processori e, quando necessario, migrare da un
processore
all'altro. Sono ovviamente possibili anche soluzioni intermedie; si possono
infatti creare \emph{clusters} di processori indipendenti dal resto del sistema.

Nel presente piano di ricerca si propone da un lato di estendere ulteriormente
l'implementazione dell'algoritmo di scheduling in modo da aumentarne
l'efficienza, dall'altro uno studio approfondito delle performance dello stesso
considerando tutta la gamma delle piattaforme hardware e delle applicazioni
software che ne potrebbero beneficiare. Sarà inoltre interessante analizzare
come il meccanismo di interesse si inserisca all'interno di sistemi soft
real-time basati su \emph{reservation} con condivisione di risorse (ad esempio
FRESCOR~\cite{FRESCOR} e AQuoSA~\cite{AQuoSA}).

Il programma verrà realizzato secondo le seguenti fasi:
\begin{itemize}
\item Studio preliminare per evidenziare i problemi di scalabilità che la
gestione attuale dei processi potrebbe incontrare; conseguentemente, ricerca
e studio di strutture dati utilizzabili per risolvere tali problemi. Allo
studio preliminare apparterrà anche l'individuazione dei sistemi soft
real-time che potranno permettere una valutazione comparativa delle scelte
implementative possibili e una stima reale degli overhead che i meccanismi
di interesse introducono nel sistema.
\item Valutazione delle possibili implementazioni, scelta e realizzazione
di quella che garantisce maggiore efficienza, migliore scalabilità e minori
overheads.
\item Verifica pratica della soluzione proposta ed implementata.
\end{itemize}

\section*{Studio preliminare}
Con lo studio preliminare si cercherà in un primo momento di individuare le
eventuali criticità dell'attuale implementazione dei meccanismi di scheduling
sopra descritti. A tal fine sarà interessante muoversi secondo due strade
ortogonali. Lo spazio delle moderne architetture multiprocessore è molto vasto,
sarà quindi interessante vedere come il sistema si comporta in ambienti diversi
da quello di sviluppo (ad esempio architetture SMP con elevato numero di
processori, NUMA o dispositivi embedded). Un aspetto caratterizzante una
specifica architettura come, ad esempio, la gerarchia delle caches dei
processori può influire infatti in maniera notevole sulla efficacia delle
decisioni di scheduling.
Fissando un tipo di hardware specifico
potrà poi essere interessante variare il tipo di applicazione software in modo
da tenere in considerazione il più alto numero di utilizzi possibili.

Il metodo di collezione dei dati ed analisi dei risultati delle prove
sperimentali sarà esso stesso oggetto di studio, in quanto non esistono al
momento attuale pratiche e strumenti universalmente considerati utili ed
attendibili. Uno dei problemi di maggiore rilevanza è infatti come estrarre
informazioni sul funzionamento del sistema senza influire sul funzionamento
stesso. Diversi approcci sono possibili, dalle più invasive modifiche al kernel
per collezionare i dati da spazio utente, alla meno pesante raccolta di
informazioni tramite lettura dei \emph{performance counters} (registri speciali
votati al controllo delle performance) presenti in quasi tutti i moderni
processori.

Una volta raccolti un numero sufficiente di dati sarà possibile ricercare
strutture dati che possano risolvere i problemi resi evidenti dall'analisi
sperimentale. Una valutazione comparativa a livello teorico tra tutte le
soluzioni possibili sarà necessaria già a questo stadio, in modo da evitare
la scelta di una implementazione difficilmente realizzabile in pratica. 

\section*{Progetto e realizzazione}
Si conta, in questa fase, di implementare una o più delle strutture dati
analizzate nella fase precedente. Anche se la fase di studio preliminare
avesse ristretto
la scelta ad una sola realizzazione sarà interessante implementarne diverse,
anche solo come prototipi, in modo da poter effettivamente dimostrare la bontà
della scelta fatta. Se, in alternativa, i dati sperimentali andassero contro
quanto sostenuto dalla teoria, questo necessariamente porterebbe sia ad una
revisione del modello teorico che alla modifica delle scelte implementative.

Nella fase di progetto prima e di realizzazione poi sarà importante anche
tener conto della possibilità di integrazione con sistemi volti a rendere
disponibili, all'interno del kernel Linux, meccanismi per il controllo della
qualità del servizio (ancora come esempio, AQuoSA). Non è ancora chiaro se
l'introduzione di tutte le modifiche descritte pregiudichi in qualche modo
il corretto funzionamento dei framework esistenti; studiare inoltre quali
benefici questi possano trarre dai nuovi meccanismi, progettare e quindi
realizzare gli adattamenti necessari per una efficace integrazione sarà quindi
parte molto interessante dell'attività di ricerca. 

Infine, visto che nelle applicazioni multithread il problema della gestione
delle risorse condivise è di particolare interesse, sarà necessario considerare
come tale problema debba essere affrontato dai meccanismi di scheduling
implementati. Si pensa quindi che possa essere necessario progettare la
realizzazione di strumenti che rendano il processo di scheduling conscio
dell'esistenza di tali risorse, in modo che esso possa prendere decisioni
che tengano conto delle problematiche relative.

\section*{Verifica dei risultati}
In questa ultima fase tutto quanto è stato realizzato sarà oggetto di
verifica. Gli stessi test sul sistema effettuati nella fase di studio
preliminare saranno adesso ripetuti per accertare la validità delle scelte
implementative. Una volta ottenuti risultati significativi, questi potranno
anche essere messi a confronto con quanto ottenuto in ricerche indipendenti
(si vedano ad esempio~\cite{H-EDF, imp_global_sched, scalability_sched} )
in modo da poterne confermare o confutare le conclusioni.

Assumendo che il sistema abbia guadagnato in termini di performance globale,
sarà poi interessante valutare se questo si comporti realmente secondo quanto
stabilisce la teoria dello scheduling EDF globale. A tal fine potranno essere
utilizzati strumenti già esistenti~\cite{unit_trace} o in alternativa si
provvederà ad implementarne di nuovi.

Ultimo test interessante, anche in vista di possibili sviluppi futuri della
ricerca sull'argomento, sarà analizzare come il sistema si comporti quando
coeasistano all'interno di esso processi assegnati staticamente ad alcuni
processori e altri liberi invece di spostarsi su tutti i processori disponibili.

\bibliography{bibliography}
\bibliographystyle{plain}
\end{document}
