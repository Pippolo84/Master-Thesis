\chapter{PRACTISE framework\label{chap:PRACTISE}}

In this chapter we will describe PRACTISE, a novel framework to help developing
new scheduling algorithm for the Linux kernel in user space. 
We briefly present a survey about the state-of-art kernel development tools,
highlighting the major advantages and drawbacks of each one. Then, we will show 
why a PRACTISE may be useful and how it is designed.

Finally, we will compare the results of some experiments made both in PRACTISE 
and in the Linux kernel.

\section{Tools for Linux kernel development\label{kernel_dev_tools}}
Scheduling on multi-core and multiprocessor system is an open research field
both from the point of view of the theory and for the technical difficulties in
implementing an efficient scheduling algorithm in the kernel.

Regarding the second problem, we're going to point out the difficulties that kernel
developers encounter in their task.

The scheduler is a fundamental part of the operating system kernel: a buggy scheduler
will soon chrash the system, usually at random and unexpected points. The major difficulty
that a prospective developer encounters when developing a new scheduling algorithms derives
from the fact that, when the system crashes, it is difficult to reconstruct the sequence
of events and states that led to the crash.

The developer has to carefully analyse system logs and traces, but this task is far
from simple due to the complexity of the kernel itself: the number of functions that compose
a commercial \emph{OS} like Linux is huge. More importantly, it is often impossible to 
impose a precise sequence of events to deterministically reproduce a particular status.
Hence, it is practically impossible to run a sequence of test-cases.

This problem is exacerbated in multi-core architectures where the scheduler service routines
run in parallel on the different processors, and make use of shared data structures that are
accessed in parallel. In these cases, it is necessary to ensure that the data structures
remain consistent under every possible interleaving of the service functions: as we will see
in the following sections, this problem is far from trivial.

Now let us present a quick list of the most widely adopted solutions for Linux kernel development,
with particular reference to the tools specifically designed for the developing of a new 
scheduling algorithm.

\subsection{LinSched\label{LinSched}}
\emph{LinSched} was originally developed by the Real Time System Group at University of
North Carolina at Chapel Hill, and it's currently mantained by P. Turner from Google. This
tool lets developers modify the behaviour of the Linux scheduler and test changes in user-space.
One of the major strength points of this tool is that it introduces very few modifications
in the kernel sources: the developer can write kernel code and, once satisfied by tests,
he has kernel ready patches at hand. One key point of LinSched is that it runs as a single
thread user-space program. This leads to a facilitated debugging process, because we can
effectively use user-space common tool like, among the others: \textit{GDB}, \textit{gprof} 
and \textit{Valgrind}.

On the other hand, single-threading is a notable drawback when we are focusing on the 
analysis of behaviour assumining a high degree of concurrency. LinSched can indeed verify 
locking, but it cannot precisely model multi-core contention.

\subsection{LITMUS$^{RT}$\label{LITMUS}}
We have already described LITMUS in Section~\ref{sec:StateArt_LITMUS}, here we are going
to point out the facilities that come with LITMUS to facilitate the development of a 
new real-time scheduling algorithm.

LITMUS provides an integrated tracing infrastructure (named \emph{Feather-Trace}) with which performance
and overhead data can be collected for off-line processing.

Being a research tool rather than a production-quality system, LITMUS does not target
Linux mainline inclusion nor POSIX-compliance: in other words code patches created with it
cannot be seamless applied to a ``Vanilla'' Linux kernel.

\subsection{KVM + GDB\label{KVM_GDB}}
The very first step after having modified the kernel is usually to run it on a virtualized
environment. This solution allows to create a virtual machine with suitable characteristics
for the developed code (like a high number of virtual cores to simulate a high concurrency
platform) and with a faster booting process compared to that of a physical machine.

In addition to this, KVM has on option to expose a server on a port where GDB can connect to
control the kernel execution. Even if this solution has some limitations, like the impossibility
of using software breakpoints, it is indeed an invaluable help in the debugging process.

Unfortunately, this solution can hardly be used in a presence of high concurrency, moreover,
it can occasionally affect the repeatability of certain bugs.

\section{PRACTISE architecture\label{PRACTISE_arch}}
PRACTISE emulates the behaviour of the Linux scheduler subsystem on a multi-core architecture
with \emph{M} parallel cores. The tool can be executed on a machine with \emph{N} cores, with
\emph{N} that can be less, equal or greater than \emph{M}. The tool can be executed in one of
the following modes:

\begin{itemize}
\item \emph{testing}
\item \emph{performance analysis}
\end{itemize}

Each processor in the simulated system is modelled by a software thread that performs a cycle
in which:

\begin{itemize}
\item scheduling events are generated at random
\item the corresponding scheduling functions are invoked
\item statistics are collected
\end{itemize}

In \emph{testing mode}, a special ``testing'' thread is executed periodically and it performs
consistency checks on the shared data structures. In the \emph{performance analysis} mode, instead,
each thread is \emph{pinned} on a processor, and the memory is locked to avoid spurious page
faults; for this reason, to obtain realistic performances it is necessary to set 
\( \emph{M} \leq \emph{N} \).

\subsection{Ready queues\label{PRACTISE_ready_queues}}
In the current version of PRACTISE the structure of distributed queues as it is in the kernel
has been mantained. The same \emph{push} and \emph{pull} algorithms used in Linux to
migrate tasks between runqueues, as described in Section~\ref{sec:MULTICORERT}, have been implemented too. 
To speed up the \emph{push} operation we have seen that the current release of \texttt{SCHED\_DEADLINE}
uses a max heap to store the deadlines of the tasks executing on the processors. 
In a similar manner, the current release of \texttt{SCHED\_RT} scheduling class uses a priority 
map\footnote{implemented in kernel/sched/cpupri.c} to record, for each processor, the priority 
value of the highest priority tasks. We find those global data structure even in
PRACTISE, with one key difference: in PRACTISE we developed and tested a \emph{cpudl} data
structure to speed up also the \emph{pull} operations in \texttt{SCHED\_DEADLINE} scheduling class. 
This solution and its potential advantages has been already described in Section~\ref{sec:pull_algo}.

During the simulation, tasks are inserted into (removed from) the ready queues using the
\texttt{enqueue()} (\texttt{dequeue()}) function, respectively. In Linux, the queues are implemented
as red-black trees. In PRACTISE, for the sake of simplicity, we have implemented them as priority heaps, 
using the data structure proposed by B. Brandenburg~\footnote{Code available here:
  \url{http://bit.ly/IozLxM}.}. Since we are mainly interested in
observing the migration tasks pattern of activity, this difference don't affect our
analysis.\\
In the following subsections, where we're going to analyze in great detail the tool internals, we will refer
to the global data structures used to speed up the push and pull operations as \texttt{push\_struct}
and \texttt{pull\_struct}, respectively.

\subsection{Locking and synchronization\label{PRACTISE_lock_and_synch}}
PRACTISE uses a range of locking and synchronization mechanisms that mimic the corresponding mechanisms
in the Linux kernel. An exhaustive list is given in Table~\ref{tab:lock_compare}. These differences
are major culprits for the slight changes needed to port code developed on the tool in the kernel, as
we will see in Section~\ref{sec:PRACTISE_port_Linux}.

It has to be noted that \texttt{wmb} and \texttt{rmb} kernel memory barriers have no corresponding operations
in user-space; therefore we have to issue a full memory barrier (\texttt{\_\_sync\_synchronize}) for every
occurence of them.

\subsection{Event generation and processing\label{sec:PRACTISE_event_gen}}
PRACTISE cannot execute or simulate a real application. Instead, each threads (that emulates a processor)
periodically generates random scheduling events according to a certain distribution, and calls the scheduler
functions. The goals of PRACTISE are to make sure that the developer can easily debug, test, compare and
evaluate real-time scheduling algorithms for multi-core processors. Therefore, we identified two main
events: \emph{activation} and \emph{blocking}.

When a task is activated, it must be inserted in one of the 
kernel queues; since such an event can cause a preemption, the scheduler is invoked, data structures are
updated, etc. Something similar happens when a task self-suspends (for example because it blocks on a
semaphore, or it suspends on a timer).

The pseudo-code for the task activation is represented in Listing~\ref{lst:task_activation}.

\begin{lstlisting}[language=C, caption={Task activation pseudo-code},
                        label={lst:task_activation}]

on_activation(task) {
	enqueue(task, local_queue);
	pull();	/* pre-schedule */
	push();	/* post-schedule*/
}

\end{lstlisting} 

The code mimics the sequence of events that are performed in the Linux code:

\begin{itemize}
\item First, the task is inserted in the local queue
\item Then, the scheduler performs a \emph{pre-schedule}, corresponding to \texttt{pull()}, which looks
at the global data structure \texttt{pull\_struct} to find the task to be pulled; if it finds it, does
a sequence of \texttt{dequeue()} and \texttt{enqueue()}.
\item Then, the Linux scheduler performs the real schedule function; this corresponds to setting the
\texttt{curr} pointer to the executing task. In PRACTISE this step is skipped, as there is no real
context switch to be performed.
\item Finally, a \emph{post-schedule} is performed, consisting of a \texttt{push()} operation, which
looks at the global data structure \texttt{push\_struct} to see if some task need to be migrated, and
in case the response is positive, performs a \texttt{dequeue()} followed by an \texttt{enqueue()}.
A similar thing happens when a task blocks (see the pseudo-code for the task blocking operation
in Listing~\ref{lst:task_blocking}).
\end{itemize}

\begin{lstlisting}[language=C, caption={Task blocking pseudo-code},
                        label={lst:task_blocking}]

on_block(task) {
        dequeue(&task, local_queue);
        pull(); /* pre-schedule */
        push(); /* post-schedule*/
}

\end{lstlisting}

\begin{table*}[htb]
\centering
\small
\renewcommand{\arraystretch}{1.5}
\begin{tabular}{|c|c|c|}
  \hline \textbf{Linux}& \textbf{PRACTISE}& \textbf{Action}\\[5pt]
  \hline raw\_spin\_lock   & pthread\_spin\_lock       & lock a structure\\
  \hline raw\_spin\_unlock & pthread\_spin\_unlock     & unlock a structure\\
  \hline atomic\_inc       & \_\_sync\_fetch\_and\_add & add a value in memory atomically\\
  \hline atomic\_dec       & \_\_sync\_fetch\_and\_sub & subtract a value in memory atomically\\
  \hline atomic\_read      & simple read               & read a value from memory \\ 
  \hline wmb               & \_\_sync\_synchronize     & issue a memory barrier \\
  \hline rmb               & \_\_sync\_synchronize     & issue a read memory barrier \\
  \hline mb                & \_\_sync\_synchronize     & issue a full memory barrier \\\hline
\end{tabular}

%%% Local Variables: 
%%% mode: latex
%%% TeX-master: "../framework-paper"
%%% End: 

\caption{Locking and synchronisation mechanisms (Linux vs. PRACTISE).}
\label{tab:lock_compare}
\end{table*}

As anticipated, every processor is simulated by a periodic thread. The thread period can be set varying
a constant in the \texttt{parameters.h} header file and represents the average frequency of events 
arriving at the processor. At every cycle, the thread randomly select one between the following events:

\begin{itemize}
\item \texttt{activation}
\item \texttt{early\_finish}
\item \texttt{idle}
\end{itemize}

In the first case, a task is generated with a random value of the deadline and function 
\texttt{on\_activation()} is called. In the second case, the task currently executing on the processor 
blocks: therefore function \texttt{on\_block()} is called. In the last case, nothing happens.
Additionally, in all cases, the deadline of the executing task is checked against the current time: 
if the deadline has passed, then the current task is blocked, and here again, function \texttt{on\_block()} 
is called.

With PRACTISE, it is possible to specify the period of the thread cycle, the probability of an activation
event, and the probability of an early finish.

\subsection{Data structures in PRACTISE\label{sec:PRACTISE_data_struct}}

PRACTISE has a modular structure, tailored to provide flexibility in developing new algorithms. The
interface exposed to the user consists of hooks to function that each global structure must provide.
The most important hooks are:

\begin{description}
\item[\texttt{data\_init}] initialises the structure, e.g. spinlock init, dynamic memory allocation, etc.
\item[\texttt{data\_cleanup}] performs clean up tasks at the end of a simulation.
\item[\texttt{data\_preempt}] called each time a local queue chenges its status (e.g. an arriving task
has higher priority that the currently executing one, so it causes a preemption); this function modifies 
the global structure to reflect new local queue status.
\item[\texttt{data\_find}] used by a scheduling policy to find the best CPU to (from) which push (pull) a
task.
\item[\texttt{data\_check}] implements the \emph{checker} mechanism (described below).
\end{description}

One of the major features provided by PRACTISE is the \emph{checking} infrastructure. Since each data
structure has to obey different rules to preserve consistency among successive updates, the user has to
equip the implemented algorithm with a proper checking function. When the tool is used in testing mode,
the \emph{data\_check} function is called at regular intervals. Therefore, an on-line validation is
performed in presence of real concurrency thus increasing the probability of discovering bugs at an
early stage of the development process. User-space debugging techniques can then be used to fix design
or developing flaws.

To give an example, the \emph{checking} function for SCHED\_DEADLINE \emph{cpudl} structure ensures 
the max-heap property: if \emph{B} is a child node of \emph{A}, then \( \emph{deadline(A)} \geq 
\emph{deadline(B)} \); it also check consistency between the heap and the array used to perform
updates on intermediates nodes (see~\cite{lelli2011efficient} for
further details). We also implemented a checking function for \emph{cpupri} data structure:
periodically, all ready queues are locked, and the content of the data structure is compared against
the corresponding highest priority task in each queue, and the consistency of the flag \texttt{overloaded}
in the \texttt{struct root\_domain} is checked. We found that the data struture id always perfectly
consistent to an external observer.

\section{Performance analysis with PRACTISE\label{PRACTISE_perf_anal}}

To collect the measurements we use the TSC\footnote{Time Stamp Counter} of IA-32 and IA-64 Instruction
Set Architectures. The TSC is a special 64-bit per-CPU register that is incremented every clock cycle.
This register can be read with two different instructions: \texttt{RDTSC} and \texttt{RDTSCP}. The latter
reads the TSC and other information about the CPU that issues the instruction itself. However, there a
number of possible issues that needs to be addressed in order to have a reliable measure:

\begin{itemize}
\item \emph{CPU frequency scaling and power management.} Modern CPUs can dynamically vary frequency to reduce
energy consumption. Recently, CPUs manufacturer has introduced a special version of TSC inside their CPUs:
\emph{constant TSC}. This kind of register is always incremented at CPU maximum frequency, regardless
of CPU actual frequency. Every CPU that supports that feature has the flag \emph{constant\_tsc} in
\texttt{/proc/cpuinfo} proc file of Linux. Unfortunately, even if the update rate of TSC is constant in
these conditions, the CPU frequency scaling can heavily alter measurements by slowing down the code
unpredictably; hence, we have conducted every experiment with all CPUs at fixed maximum frequency and
no power-saving features enabled. To do this the \texttt{cpufreq} utility has been used.
\item \emph{TSC synchronisation between different cores.} Since every core has its own TSC, it is
possible that a misalignment between different TSCs may occur. Even if the kernel runs a synchronisation
routine at start up (as we can see in the kernel log message), the synchronisation accuracy is tipically
in the range of several hundred clock cycles. To avoid this problem, we have set CPU affinity of every
thread with a specific CPU index. In other words we have a 1:1 association between threads and CPUs, fixed
for the entire simulation time. In this way we also prevent thread migration during an operation, which
may introduce unexpected delays.
\item \emph{CPU instruction reordering.} To avoid instruction reordering, we use two instructions that
guarantees serialisation: \texttt{RDTSCP} and \texttt{CPUID}. The latter guarantees that no instructions
can be moved over or beyond it, but has a non-negligible and variable calling overhead. The former, in
contrast, only guarantees that no previous instructions will be moved over. In conclusion, as suggested
in~\cite{intelTSCpaper}, we used the following sequence to measure a given code snippet:
\begin{lstlisting}
  CPUID
  RDTSC
  code
  RDTSCP 
  CPUID
\end{lstlisting}
\item \emph{Compiler instruction reordering.} Even the compiler can reorder instructions; so we marked the
inline asm code that reads and saves the TSC current value with the keyword \emph{volatile}.
\item \emph{Page faults.} To avoid page fault time accounting we locked every page of the process in memory
with a call to \texttt{mlockall}.
\end{itemize}

PRACTISE collects every measurement sample in a global multidimensional array, where we keep samples coming
from different CPUs separated. After all simulation cycles are terminated, all the samples are written to
an output file.

By default, PRACTISE measures the following statistics:

\begin{itemize}
\item duration and number of \emph{push} and \emph{pull} operations;
\item number of \emph{enqueue} and \emph{dequeue} operations;
\item duration and number of \emph{data\_preempt}, \emph{data\_finish} and \emph{data\_find} operations
\end{itemize}

It is possible to add different measures in the code of a specific algorithm by using PRACTISE's macros.\\
For example, suppose that we want to measure the number of clock cycles that a code snippet takes to be
executed: we refer to this piece of code as \emph{operation\_under\_measure} in the subsequent
explanation.

To enable the measure samples collection we have to:

\begin{itemize}
\item insert the following lines of code in \emph{include/measure.h} header file:

\begin{lstlisting}
EXTERN_MEASURE_VARIABLE(operation_under_measure)
EXTERN_DECL(ALL_COUNTER(operation_under_measure))
\end{lstlisting}

and also the following lines in \emph{src/measure.c} source file:

\begin{lstlisting}
MEASURE_VARIABLE(operation_under_measure)
ALL_COUNTER(operation_under_measure)
\end{lstlisting}

to declare the array that will holds all the measurement samples and the variable that will 
holds the number of measurement samples collected (useful for average calculation).

\item insert the following lines of code at the beginning and at the end of the \texttt{main} 
function in \emph{src/practise.c} source file, respectively:

\begin{lstlisting}
MEASURE_ALLOC_VARIABLE(operation_under_measure)

MEASURE_FREE_VARIABLE(operation_under_measure)
\end{lstlisting}

to allocate memory for the measurement samples global array.

\item surround the code snippet to measure with the following two instructions:

\begin{lstlisting}
MEASURE_START(operation_under_measure, CPU-index)

<code_to_measure>

MEASURE_END(operation_under_measure, CPU-index)
\end{lstlisting}

where \texttt{CPU-index} is the CPU that is executing the code under measure. This is
useful if we want to analyze how much operations a specific CPU carries on.

\item finally, add those lines of code in \emph{src/practise.c}:

\begin{lstlisting}
MEASURE_STREAM_OPEN(operation_to_measure, online_cpus);
for(i = 0; i < online_cpus; i++){
	MEASURE_PRINT(out_operation_to_measure, operation_to_measure, i);
	fprintf(out_operation_to_measure, "\n");
}
MEASURE_STREAM_CLOSE(operation_to_measure);
\end{lstlisting}

to print all the measurements sample in a properly formatted way.

\end{itemize}

\section{Evaluation\label{sec:PRACTISE_eval}}

In this section, we show how difficult is to port a scheduler developed with the help of
PRACTISE into the Linux kernel; then, we report performance analysis figures and discuss
the different results obtained in user space with PRACTISE and inside the kernel.

\subsection{Porting to Linux\label{sec:PRACTISE_port_Linux}}

The effort in porting an algorithm developed with PRACTISE in Linux can be estimated by
counting the number of different lines of code in the two implementations. We have two
global data structures implemented both in PRACTISE and in the Linux kernel: \emph{cpudl}
and \emph{cpupri}.\\
We used the \texttt{diff} utility to compare differences between user-space and kernel code
of each data structure. Results are summarised in Table~\ref{tab:diff}.

\begin{table*}[htb]
\centering
\small
\renewcommand{\arraystretch}{1.5}
\begin{tabular}{|c|c|c|}
  \hline \textbf{Structure} & \textbf{Modifications} & \textbf{Ratio}\\[5pt]
  \hline \emph{cpudl} & 12+ 14- & 8.2\% \\
  \hline \emph{cpupri} & 17+ 21- & 14\% \\
  \hline
\end{tabular}

%%% Local Variables: 
%%% mode: latex
%%% TeX-master: "../framework-paper"
%%% End: 

\caption{Differences between user-space and kernel code.}
\label{tab:diff}
\end{table*}

Less than 10\% of changes were required to port \emph{cpudl} to Linux, these differences
mainly due to the framework interface (that is, pointers conversion). Slightly higher 
changes ratio for \emph{cpupri}, due to the quite heavy use of atomic operations.
An example of such changes is given in Figure~\ref{fig:ex-diff} (lines with a \texttt{-}
correspond to user-space code, while those with a \texttt{+} to kernel code).

\begin{figure}
  \centering
\begin{lstlisting}[frame=single]
[...]
-void cpupri_set(void *s, int cpu, int newpri)
+void cpupri_set(struct cpupri *cp, int cpu,
+                 int newpri)
 {
-   struct cpupri *cp = (struct cpupri*) s;
    int *currpri = &cp->cpu_to_pri[cpu];
    int oldpri  = *currpri;
    int do_mb = 0;
@@ -63,57 +61,55 @@
    if (newpri == oldpri)
         return;

-   if (newpri != CPUPRI_INVALID) {
+   if (likely(newpri != CPUPRI_INVALID)) {
         struct cpupri_vec *vec =
                      &cp->pri_to_cpu[newpri];

         cpumask_set_cpu(cpu, vec->mask);
-        __sync_fetch_and_add(&vec->count, 1);
+        smp_mb__before_atomic_inc();
+        atomic_inc(&(vec)->count);
         do_mb = 1;
    }
[...]
\end{lstlisting}
  \caption{Comparison using \texttt{diff}.}
\label{fig:ex-diff}
\end{figure}

The difference on the synchronisation code can be reduced by using appropriate macros.
For example, we could introduce a macro that translates to \texttt{\_\_sync\_fetch\_and\_add}
when compiled inside PRACTISE, and to the corresponding Linux code otherwise. However,
we decide to mantain the different code to highlight the differences between the two 
frameworks, rather than hide them.

In conclusion, the amount of work shouldered on the developer to transfer the implemented
algorithm to the kernel, after testing, is quite low reducing the probability of introducing
bugs during the porting.
