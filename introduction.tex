\chapter{Introduction\label{chap:introduction}}

Multiprocessor systems are nowadays de facto standard for both personal
computers and server workstations. Benefits of dual-core and quad-core
technology is also common in embedded devices and cellular phones as
well. In fact, raw increases in computational power is no more the answer 
for overall better performance: the energy efficiency is a 
primary concern, that can't be ignored at any level of a system design, 
from hardware to software. Regarding the hardware layer, multicore and multiprocessors 
technologies surely gived an answer to that issue, but without a proper software 
design, the scalability of the entire system may suffer.

The role of the operating system scheduler is fundamental while managing
the threads of execution: a sub-optimal schedule may lead to high latency
and very poor overall performance. If real time tasks, characterized by 
strictly timing constraints, are also considered, we can easily understand 
that finding an optimal schedule is far from trivial.

Linux, as a General Purpose Operating System (GPOS), should be able
to run on every possible system, from workstations to mobile devices.
Even if each configuration has its own issues, the common trend seems
to be a considerable interest in using Linux for real-time and control
applications.

But Linux has not been designed to be a Real-Time Operating System (RTOS)
and this imply that a classical real-time feasibility study of the
system under development is not possible, there's no way to be sure that
timing requirements of tasks will be met under \emph{every} circumstance.
POSIX-compliant fixed-prority policies offered by Linux are not enough
for specific application requirements.

Great issues arise when size, processing power, energy consumption and
costs are tightly constrained. Time-sensitive applications (e.g., MPEG
players) for embedded devices have to efficiently make use of system
resources and, at the same time, meet the real-time requirements.

In a recent paper~\cite{SCHEDDEAD}, Dario Faggioli and others proposed an
implementation of the ``Earliest Deadline First'' (EDF) algorithm
in the Linux kernel. In order to extend stock Linux kernel's features
a new scheduling policy has been created: \texttt{SCHED\_DEADLINE}. Later,
Juri Lelli extended that scheduling policy to add processes migration
between CPUs ~\cite{lelli2011efficient}. This allowed to reach full utilization of the
system in multicore and multiprocessor environment. While the proposed
implementation is indeed effective, a problem of scalability arises
when the scheduler has to dinamically assigns real-time tasks to
an high number of online CPUs. All the scheduler shared data structures are
potential performance bottlenecks: the contention to manipulate tha data 
structure may increase a lot, leading to unpredictable and unbounded 
latencies.

Unfortunately, the development of new solutions to manage concurrency
in kernel space is far from trivial: when the number of parallel
scheduler instances increases the common tools used for debugging
are not so effective.

In this thesis, we propose PRACTISE, a tool for performance analysis and
testing of real-time multicore schedulers for the Linux kernel. PRACTISE
enables fast prototyping of real-time multicore scheduling mechanisms,
allowing easy debugging and testing of such mechanisms in user-space.
Thanks to PRACTISE we developed a set of innovative solutions to improve
the scalability of the processes migration mechanism. We will show that,
with those modifications, not only a better scalability has been reached,
but also a schedule closer to \emph{G-EDF} policy of the tasks has been
achieved.

This document is organized as follows.

Chapter 1 (\textbf{Background}) gives a brief overview of the concepts and
the theory on which this thesis is based. First, the modular framework of
the Linux scheduler is analyzed (with special attention to multiprocessors
systems), then we find the state of the art of real time scheduling on Linux.
Since we will improve the \texttt{SCHED\_DEADLINE} implementation, in this
chapter we also give some insights on the theory behind those real time
scheduling algorithms and analyze how they are implemented inside the
Linux kernel. Finally, we will discuss in great detail about the
current implementation of the task migrations algorithm in \texttt{SCHED\_DEADLINE}
scheduling class.


Chapter 2 (\textbf{Synchronization Mechanisms Analysis}) gives a detailed
explanation of the available mechanisms to manage concurrent accesses on
a shared data structure. In particular, we will refer to the synchronization
techniques in Linux kernel. Finally, we will explain a recently proposed
framework that aims to improve performance for shared data structures
accessed in parallel by a significant number of threads.


In Chapter 3 (\textbf{New Solutions for Task Migration}) we present a set
of new solutions for the task migration algorithms. We will show the main idea
behind each of those to explain why such a design was chosen.


Chapter 4 (\textbf{PRACTISE Framework}) contains the details of PRACTISE
implementation. Here we will explain how PRACTISE was designed and how it
can be used to facilitate the development of new kernel code. In the last
part of the chapter we focus on the ability of PRACTISE to predict the
relative performance of the various algorithm simulated with it.


Chapter 5 (\textbf{Experimental Results}) contains the graphs that show
the results of our experiments conducted with the Linux kernel. We present
the results of each new algorithm discussed above, explaining why a certain
performance trend is achieved. Doing so, we will point out the main advantages
and also the disadvantages of each solution.


Finally, in Chapter 6 (\textbf{Conclusions and Future Works}), we sum up
results and suggest possible future extensions to the code as well as alternate
ways of testing.
