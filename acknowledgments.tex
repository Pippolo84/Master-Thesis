\chapter{Acknowledgments}

Vorrei innanzitutto ringraziare il prof. Giuseppe Lipari per avermi dato
l'occasione di svolgere questa tesi. Il suo continuo supporto \`e stato un
fondamentale aiuto per raggiungere il risultato finale.

Desidero ringraziare anche il prof. Paolo Ancilotti per la sua disponibilit\`a
nell'assistermi durante la discussione.

Un ringraziamento particolare va a Juri Lelli. Juri mi ha assistito per
tutta la durata del lavoro, aiutandomi in ogni aspetto tecnico della tesi:
senza la sua grande disponibilit\`a probabilmente non potrei scrivere adesso 
queste parole.

E ora, \`e il momento di ringraziare famiglia ed amici.

Il primo pensiero \`e ``condiviso'' tra mia nonna e Katia. La prima perch\`e
in tutti questi anni non ha mai rinunciato al difficile compito di farmi,
oltre che da nonna, da mamma. Nonna, posso solo dirti che ci sei riuscita
in pieno. La seconda perch\`e ha sempre creduto in me, infodendomi
coraggio, fiducia e serenit\`a anche nei momenti più difficili. Sapere di
averti accanto \`e bellissimo: GRAZIE!
Vorrei inoltre ringraziare mio padre, che ha avuto la pazienza di aspettare
fino ad oggi: spero finalmente di averlo fatto contento con il conseguimento 
del titolo.

Poi, non posso dimenticarmi dei miei amici. Con molti di loro ho avuto
il piacere di studiare insieme: se posso ricordare con un sorriso tutti
i giorni passati in biblioteche ed aule studio, \`e sicuramente grazie a
loro. Fra questi voglio citare il Ture che prima o poi, ne sono sicuro,
riuscir\`a a vendermi una lampada (ma non riuscir\`a mai a battermi ad Hobosoccer);
Enri che \`e il secondo miglior giocatore del mondo di Hobosoccer (complimenti, 
perch\`e il primo \`e di un altro pianeta);
il Tolo, che invece a studiare al CNR non \`e mai venuto; Jacopo, che
condivide con me l'idea del giusto ``metodo'' di studio; Rani, che prima
o poi smetter\`a di usare Windows e di fare pompaggio in panca (o forse no); 
il Fao, costantemente impegnato con
i suoi ``cantieri'' ma che non perde mai occasione per dirmi che sono bravo 
e il Landi, che mi ha insegnato (e mi insegna) tantissimo su tematiche 
extra-universitarie di indubbio interesse.
Per finire l'elenco degli uomini ``duri e puri'' non posso non citare Roberta, che voglio
ringraziare per non avermi usato come merenda durante i pomeriggi di studio.
So che la tentazione \`e stata forte.
Ma un gruppo di soli maschi non fa sugo, e allora ne approfitto per ringraziare
le bellissime del Putignano's Group: Sara l'architetto dal capello corto, 
Serena che devo ringraziare poco perch\`e siamo ``nemici'', Guenda che non riuscir\`a
mai a condurre un mezzo a motore, Francesca che con la guida \`e
messa forse peggio di Guenda, Marghe che probabilmente le batte entrambe, Laura 
che credo non abbia nemmeno la patente ed altre che sicuramente adesso
non ricordo (ma nessuna di queste \`e comunque in grado di guidare).
Infine devo ringraziare Luca, un preziosissimo compagno di studi che ho avuto
il piacere di lasciare indietro (in modo imbarazzante, direi) durante un giro
in moto all'Abetone. Citarlo qui \`e l'ultima risorsa che ho per suscitare il
suo orgoglio e convincerlo a girare ancora con me.
